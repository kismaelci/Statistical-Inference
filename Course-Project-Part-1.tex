% Options for packages loaded elsewhere
\PassOptionsToPackage{unicode}{hyperref}
\PassOptionsToPackage{hyphens}{url}
%
\documentclass[
]{article}
\usepackage{amsmath,amssymb}
\usepackage{lmodern}
\usepackage{iftex}
\ifPDFTeX
  \usepackage[T1]{fontenc}
  \usepackage[utf8]{inputenc}
  \usepackage{textcomp} % provide euro and other symbols
\else % if luatex or xetex
  \usepackage{unicode-math}
  \defaultfontfeatures{Scale=MatchLowercase}
  \defaultfontfeatures[\rmfamily]{Ligatures=TeX,Scale=1}
\fi
% Use upquote if available, for straight quotes in verbatim environments
\IfFileExists{upquote.sty}{\usepackage{upquote}}{}
\IfFileExists{microtype.sty}{% use microtype if available
  \usepackage[]{microtype}
  \UseMicrotypeSet[protrusion]{basicmath} % disable protrusion for tt fonts
}{}
\makeatletter
\@ifundefined{KOMAClassName}{% if non-KOMA class
  \IfFileExists{parskip.sty}{%
    \usepackage{parskip}
  }{% else
    \setlength{\parindent}{0pt}
    \setlength{\parskip}{6pt plus 2pt minus 1pt}}
}{% if KOMA class
  \KOMAoptions{parskip=half}}
\makeatother
\usepackage{xcolor}
\usepackage[margin=1in]{geometry}
\usepackage{color}
\usepackage{fancyvrb}
\newcommand{\VerbBar}{|}
\newcommand{\VERB}{\Verb[commandchars=\\\{\}]}
\DefineVerbatimEnvironment{Highlighting}{Verbatim}{commandchars=\\\{\}}
% Add ',fontsize=\small' for more characters per line
\usepackage{framed}
\definecolor{shadecolor}{RGB}{248,248,248}
\newenvironment{Shaded}{\begin{snugshade}}{\end{snugshade}}
\newcommand{\AlertTok}[1]{\textcolor[rgb]{0.94,0.16,0.16}{#1}}
\newcommand{\AnnotationTok}[1]{\textcolor[rgb]{0.56,0.35,0.01}{\textbf{\textit{#1}}}}
\newcommand{\AttributeTok}[1]{\textcolor[rgb]{0.77,0.63,0.00}{#1}}
\newcommand{\BaseNTok}[1]{\textcolor[rgb]{0.00,0.00,0.81}{#1}}
\newcommand{\BuiltInTok}[1]{#1}
\newcommand{\CharTok}[1]{\textcolor[rgb]{0.31,0.60,0.02}{#1}}
\newcommand{\CommentTok}[1]{\textcolor[rgb]{0.56,0.35,0.01}{\textit{#1}}}
\newcommand{\CommentVarTok}[1]{\textcolor[rgb]{0.56,0.35,0.01}{\textbf{\textit{#1}}}}
\newcommand{\ConstantTok}[1]{\textcolor[rgb]{0.00,0.00,0.00}{#1}}
\newcommand{\ControlFlowTok}[1]{\textcolor[rgb]{0.13,0.29,0.53}{\textbf{#1}}}
\newcommand{\DataTypeTok}[1]{\textcolor[rgb]{0.13,0.29,0.53}{#1}}
\newcommand{\DecValTok}[1]{\textcolor[rgb]{0.00,0.00,0.81}{#1}}
\newcommand{\DocumentationTok}[1]{\textcolor[rgb]{0.56,0.35,0.01}{\textbf{\textit{#1}}}}
\newcommand{\ErrorTok}[1]{\textcolor[rgb]{0.64,0.00,0.00}{\textbf{#1}}}
\newcommand{\ExtensionTok}[1]{#1}
\newcommand{\FloatTok}[1]{\textcolor[rgb]{0.00,0.00,0.81}{#1}}
\newcommand{\FunctionTok}[1]{\textcolor[rgb]{0.00,0.00,0.00}{#1}}
\newcommand{\ImportTok}[1]{#1}
\newcommand{\InformationTok}[1]{\textcolor[rgb]{0.56,0.35,0.01}{\textbf{\textit{#1}}}}
\newcommand{\KeywordTok}[1]{\textcolor[rgb]{0.13,0.29,0.53}{\textbf{#1}}}
\newcommand{\NormalTok}[1]{#1}
\newcommand{\OperatorTok}[1]{\textcolor[rgb]{0.81,0.36,0.00}{\textbf{#1}}}
\newcommand{\OtherTok}[1]{\textcolor[rgb]{0.56,0.35,0.01}{#1}}
\newcommand{\PreprocessorTok}[1]{\textcolor[rgb]{0.56,0.35,0.01}{\textit{#1}}}
\newcommand{\RegionMarkerTok}[1]{#1}
\newcommand{\SpecialCharTok}[1]{\textcolor[rgb]{0.00,0.00,0.00}{#1}}
\newcommand{\SpecialStringTok}[1]{\textcolor[rgb]{0.31,0.60,0.02}{#1}}
\newcommand{\StringTok}[1]{\textcolor[rgb]{0.31,0.60,0.02}{#1}}
\newcommand{\VariableTok}[1]{\textcolor[rgb]{0.00,0.00,0.00}{#1}}
\newcommand{\VerbatimStringTok}[1]{\textcolor[rgb]{0.31,0.60,0.02}{#1}}
\newcommand{\WarningTok}[1]{\textcolor[rgb]{0.56,0.35,0.01}{\textbf{\textit{#1}}}}
\usepackage{graphicx}
\makeatletter
\def\maxwidth{\ifdim\Gin@nat@width>\linewidth\linewidth\else\Gin@nat@width\fi}
\def\maxheight{\ifdim\Gin@nat@height>\textheight\textheight\else\Gin@nat@height\fi}
\makeatother
% Scale images if necessary, so that they will not overflow the page
% margins by default, and it is still possible to overwrite the defaults
% using explicit options in \includegraphics[width, height, ...]{}
\setkeys{Gin}{width=\maxwidth,height=\maxheight,keepaspectratio}
% Set default figure placement to htbp
\makeatletter
\def\fps@figure{htbp}
\makeatother
\setlength{\emergencystretch}{3em} % prevent overfull lines
\providecommand{\tightlist}{%
  \setlength{\itemsep}{0pt}\setlength{\parskip}{0pt}}
\setcounter{secnumdepth}{-\maxdimen} % remove section numbering
\ifLuaTeX
  \usepackage{selnolig}  % disable illegal ligatures
\fi
\IfFileExists{bookmark.sty}{\usepackage{bookmark}}{\usepackage{hyperref}}
\IfFileExists{xurl.sty}{\usepackage{xurl}}{} % add URL line breaks if available
\urlstyle{same} % disable monospaced font for URLs
\hypersetup{
  pdftitle={investigate the exponential distribution in R and compare it with the Central Limit Theorem},
  pdfauthor={ismael},
  hidelinks,
  pdfcreator={LaTeX via pandoc}}

\title{investigate the exponential distribution in R and compare it with
the Central Limit Theorem}
\author{ismael}
\date{2022-09-21}

\begin{document}
\maketitle

\hypertarget{overview}{%
\subsection{Overview}\label{overview}}

In this project you will investigate the exponential distribution in R
and compare it with the Central Limit Theorem. The exponential
distribution can be simulated in R with rexp(n, lambda) where lambda is
the rate parameter. The mean of exponential distribution is 1/lambda and
the standard deviation is also 1/lambda. Set lambda = 0.2 for all of the
simulations. You will investigate the distribution of averages of 40
exponentials. Note that you will need to do a thousand simulations.

\begin{verbatim}
## Warning: package 'knitr' was built under R version 4.2.1
\end{verbatim}

\begin{verbatim}
## Warning: package 'ggplot2' was built under R version 4.2.1
\end{verbatim}

\begin{verbatim}
## Warning: package 'stringr' was built under R version 4.2.1
\end{verbatim}

\begin{verbatim}
## Warning: package 'tidyverse' was built under R version 4.2.1
\end{verbatim}

\begin{verbatim}
## -- Attaching packages --------------------------------------- tidyverse 1.3.2 --
## v tibble  3.1.8      v purrr   0.3.4 
## v tidyr   1.2.1      v dplyr   1.0.10
## v readr   2.1.2      v forcats 0.5.2
\end{verbatim}

\begin{verbatim}
## Warning: package 'tibble' was built under R version 4.2.1
\end{verbatim}

\begin{verbatim}
## Warning: package 'tidyr' was built under R version 4.2.1
\end{verbatim}

\begin{verbatim}
## Warning: package 'readr' was built under R version 4.2.1
\end{verbatim}

\begin{verbatim}
## Warning: package 'dplyr' was built under R version 4.2.1
\end{verbatim}

\begin{verbatim}
## Warning: package 'forcats' was built under R version 4.2.1
\end{verbatim}

\begin{verbatim}
## -- Conflicts ------------------------------------------ tidyverse_conflicts() --
## x dplyr::filter() masks stats::filter()
## x dplyr::lag()    masks stats::lag()
\end{verbatim}

\begin{verbatim}
## Warning: package 'rmarkdown' was built under R version 4.2.1
\end{verbatim}

\hypertarget{simulations}{%
\subsection{Simulations}\label{simulations}}

Illustrate via simulation and associated explanatory text the properties
of the distribution of the mean of 40 exponentials

Run Simulations with variables

\hypertarget{calculate-sample-mean}{%
\paragraph{Calculate Sample Mean}\label{calculate-sample-mean}}

Calculating the mean from the simulations with give the sample mean.

\begin{verbatim}
## [1] 4.990025
\end{verbatim}

\hypertarget{calculate-theoretical-mean}{%
\paragraph{Calculate Theoretical
Mean}\label{calculate-theoretical-mean}}

Calculating the theoretical mean of an exponential distribution
(lambda\^{}-1).

\begin{verbatim}
## [1] 5
\end{verbatim}

\hypertarget{comparison-of-means}{%
\paragraph{Comparison of means}\label{comparison-of-means}}

Sample Mean and Theoretical Mean are almost the same

\begin{verbatim}
## [1] 0.009974799
\end{verbatim}

\hypertarget{calculate-sample-variance}{%
\paragraph{Calculate Sample Variance}\label{calculate-sample-variance}}

Calculating the variance from the simulation means with give the sample
variance.

\begin{verbatim}
## [1] 0.6111165
\end{verbatim}

\hypertarget{calculate-theoretical-variance}{%
\paragraph{Calculate Theoretical
Variance}\label{calculate-theoretical-variance}}

The theoretical variance of an exponential distribution is (lambda *
sqrt(n))\^{}-2.

\begin{verbatim}
## [1] 0.625
\end{verbatim}

\hypertarget{comparison-of-variances}{%
\paragraph{Comparison of Variances}\label{comparison-of-variances}}

Sample Variance and Theoretical Variance are almost the same

\begin{verbatim}
## [1] 0.01388353
\end{verbatim}

\hypertarget{distribution}{%
\subsection{Distribution}\label{distribution}}

\begin{Shaded}
\begin{Highlighting}[]
\NormalTok{simnum }\OtherTok{\textless{}{-}} \DecValTok{1000}
\NormalTok{sim }\OtherTok{\textless{}{-}} \FunctionTok{matrix}\NormalTok{(}\FunctionTok{rexp}\NormalTok{(simnum}\SpecialCharTok{*}\NormalTok{exponentials, }\AttributeTok{rate=}\NormalTok{lambda), simnum, exponentials)}
\NormalTok{row\_means }\OtherTok{\textless{}{-}} \FunctionTok{rowMeans}\NormalTok{(sim)}
\CommentTok{\# plot of histogram of means}
\FunctionTok{hist}\NormalTok{(row\_means, }\AttributeTok{breaks=}\DecValTok{100}\NormalTok{, }\AttributeTok{prob=}\ConstantTok{TRUE}\NormalTok{,}
     \AttributeTok{main=}\StringTok{"Normal Distribution Comparison"}\NormalTok{, }\AttributeTok{xlab=}\StringTok{"Sample mean"}\NormalTok{)}
\CommentTok{\# density of the means of samples}
\FunctionTok{lines}\NormalTok{(}\FunctionTok{density}\NormalTok{(row\_means), }\AttributeTok{col=}\StringTok{"blue"}\NormalTok{)}
\CommentTok{\# theoretical center of distribution}
\FunctionTok{abline}\NormalTok{(}\AttributeTok{v=}\DecValTok{1}\SpecialCharTok{/}\NormalTok{lambda, }\AttributeTok{col=}\StringTok{"red"}\NormalTok{)}
\CommentTok{\# theoretical density of the means of samples}
\NormalTok{xfit }\OtherTok{\textless{}{-}} \FunctionTok{seq}\NormalTok{(}\FunctionTok{min}\NormalTok{(row\_means), }\FunctionTok{max}\NormalTok{(row\_means), }\AttributeTok{length=}\DecValTok{100}\NormalTok{)}
\NormalTok{yfit }\OtherTok{\textless{}{-}} \FunctionTok{dnorm}\NormalTok{(xfit, }\AttributeTok{mean=}\DecValTok{1}\SpecialCharTok{/}\NormalTok{lambda, }\AttributeTok{sd=}\NormalTok{(}\DecValTok{1}\SpecialCharTok{/}\NormalTok{lambda}\SpecialCharTok{/}\FunctionTok{sqrt}\NormalTok{(exponentials)))}
\FunctionTok{lines}\NormalTok{(xfit, yfit, }\AttributeTok{pch=}\DecValTok{22}\NormalTok{, }\AttributeTok{col=}\StringTok{"green"}\NormalTok{, }\AttributeTok{lty=}\DecValTok{2}\NormalTok{)}
\CommentTok{\# add legend}
\FunctionTok{legend}\NormalTok{(}\StringTok{\textquotesingle{}bottomright\textquotesingle{}}\NormalTok{, }\FunctionTok{c}\NormalTok{(}\StringTok{"simulation"}\NormalTok{, }\StringTok{"theoretical"}\NormalTok{), }\AttributeTok{lty=}\FunctionTok{c}\NormalTok{(}\DecValTok{1}\NormalTok{,}\DecValTok{2}\NormalTok{), }\AttributeTok{col=}\FunctionTok{c}\NormalTok{(}\StringTok{"blue"}\NormalTok{, }\StringTok{"green"}\NormalTok{))}
\end{Highlighting}
\end{Shaded}

\includegraphics{Course-Project-Part-1_files/figure-latex/unnamed-chunk-8-1.pdf}

Due to the Central Limit Theorem, the distribution of means of the
sampled exponential distributions follow a normal distribution.

\end{document}
